\documentclass[a4paper,11pt]{article}

\usepackage{exptech}
\linespread{1,6}

%%%%%%%%%%%%%%%%%%%%%%%%%%%%%%%%%%%%%%%%%%%%%%%%%%%%%%%%%%%%%%%%%%%%%%%%%%%%%%%

\title{ \textbf{Création d'un modèle 3D à partir de dessins 2D} }
% Pour avoir le titre de l'expose sur chaque page

\author{ Aurélien \textsc{FONTAINE} Etienne \textsc{GEANTET} \\
	Manutea \textsc{HUANG} Arnaud \textsc{MARTIN} \\
	\\
	Encadrants : François \textsc{LEHERICEY}	Bertrand \textsc{COUASNON}}

\date{4 Mai 2015}                    % Ne pas modifier

%%%%%%%%%%%%%%%%%%%%%%%%%%%%%%%%%%%%%%%%%%%%%%%%%%%%%%%%%%%%%%%%%%%%%%%%%%%%%%%

\begin{document}

\maketitle                 % Génère le titre
\thispagestyle{empty}      % Supprime le numéro de page sur la 1re page

\begin{abstract}
	Lors de leur troisième année, les étudiants de l'INSA de Rennes doivent réaliser un projet. Notre groupe a choisi de travailler sur le développement d'une application sur tablette. Cette application doit permettre à un utilisateur novice de créer rapidement et facilement des objets en 3D à partir d'un dessin 2D. 
\end{abstract}
	
	\section{Remerciements}
		Nous souhaitons tout d'abord remercier Monsieur François LEHERICEY pour sa disponibilité ainsi que pour ses précieuses informations.
		
		Nous voudrions également remercier Monsieur Bertrand COUASNON pour son apport de connaissances ainsi que son aide dans l'élaboration de nos différentes présentations.
				
	\section{Introduction} %Pas oublier de changer GRIBOUILLI
		%Présentation succinte du projet
		Notre équipe s'est mobilisée autour d'une question : Comment un utilisateur novice peut-il créer facilement et rapidement un objet en 3D à partir d'un gribouilli ?
		
		Afin d'y répondre nous avons cherchés à développer une application intuitive. Mais nous avions quelques contraintes : l'utilisateur dessinera sur la tablette et les objets 3D devront être exportés vers un serveur Unity. Cet envoi à un serveur Unity doit permettre aux chercheurs de l'IRISA d'insérer rapidement des objets simples dans une de leurs scènes.
	\section{État de l'art}
		%Rappel rapide des technologies utilisées
	\section{Organisation du travail}
	%découpage des tâches et répartition du travail + explication séances hebdomadaires
	Dès le début du projet, nous nous sommes divisés le travail afin que chacun puisse avancer sans pour autant devoir attendre le travail d'un autre membre de notre équipe. Nous avons repéré les axes centraux du développement qui sont : l'affichage de la fenêtre de l'application, la capture et l'affichage du dessin fait par l'utilisateur, l'extrusion de la forme dessinée, le placement de celle-ci, et enfin l'exportation de la figure finale sur un serveur Unity. Ceux-ci ont donc été partagés entre nous.
	
	Afin de faire un point régulier et d'être sûr de répondre aux exigences de notre encadrant, nous nous sommes réunis hebdomadairement. Au cours de ces séances, nous pouvions ainsi voir le travail effectué par les différents membres de notre projet et nous pouvions bénéficier de l'expérience de M. François LEHERICEY avec Unity pour les problèmes dont on ne trouvait pas les solutions.
	\section{Présentation de l'application}
		%combien de temps passé pour chaque tâche
		%Problèmes rencontrées et solutions apportées, en quoi ces solutions ont été utilisées par la suite
		\subsection{L'IHM}
			Nous devions d'abord créer le design de l'Interface Homme-Machine (HIM). Nous avons défini sur papier auquel devaient ressembler les divers menus. Afin de les mettre en place sous Unity, nous avons dû nous servir de leur mise à jour 4.6 qui contient un système de fenêtre adapté à la création de menus (UI), sortie fin novembre 2014. Au vu de sa date de sortie récente, nous avions accès à peu d'aides sur Internet. Après une avancée assez laborieuse sur le début, nous avons réussi à mettre en place un canva...(line truncated)...
			%expliquer ce qu'est un canvas ?
			Le temps passé sur cette partie, nous avons permis d'acquérir les connaissances nécessaires sur les outils UI d'Unity et donc d'avancer plus rapidement dans la suite du développement. De plus les différents boutons étaient en place et n'attendaient plus que les scripts écrits dans les autres parties.
		
		\subsection{Le dessin}
			La capture du dessin étant une partie lourde, nous avons décidé de séparer les outils de cette partie. C'est la personne en charge de l'IHM, donc déjà accoutumée avec les menus qui s'en sont occupés.
		
		
			\subsubsection{Les outils}				
				Suite directe de l'IHM, un sous-menu coulissant devait être présent pour pouvoir choisir les outils de dessin. Pour rendre le menu coulissant, la solution qui nous a paru la plus adaptée est de se servir de la gestion des animations d'Unity. Une fois celles-ci faites, elles ont été liées à des boutons pour les enclencher.
					
				Dans cette partie la difficulté était de choisir comment coder la sélection d'outils. Pour cela, nous avons choisi de créer une énumération pour chaque ensemble d'outils (couleurs, diamètres, pinceau/gomme/pot de peinture). Des fonctions pour récupérer les outils courants ont été implantées, pour permettre au reste de l'application d'avoir connaissance de l'état des outils. De même des fonctions de Set ont été liées aux boutons appropriés. Pour respecter les droits d'images, nous avons dessiné nous-même...(line truncated)...
				
				Cette partie n'était pas très compliquée en elle-même. Après des tests utilisateurs, nous avons accéléré les animations. Et au cours du développement, ces animations ont dû être plusieurs fois refaites, car elles déclenchaient des bugs non expliqués.
			\subsubsection{La zone de dessin}
			
			C'est pour nous la première étape de la création d'un objet 3D. A la suite de cette étape, nous souhaitions obtenir une image à extruder. Les images sont appelées textures sous Unity. Nous avons utilisé un plan comme object de référence pour la zone de dessin.
			
			Un premier problème apparut alors quant au dimensionnement de cette zone de dessin. Le plan ayant une taille précise, il fallait être capable de modifier sa taille par rapport à celle du canvas. Nous avons donc rédigé un script qui initialise la dimension de la zone de dessin, autant pour le plan que pour la texture qui lui est associée.
		\subsection{L'extrusion}
		\subsection{Placement de la figure}
		\subsection{La gestion de la liste des figures}
		\subsection{L'envoi}
	\section{Objectifs}
		%Nos objectifs ont-il été atteints? Si Non, quels ont été les choix faits en cours de routes pour changer ou abandonner certains objectifs.
	\section{Conclusion}
		%Ce que l'on se souviendra du projet
		%Si c'était à refaire, quoi changer?
		%Quelles sont les évolutions que l'on pourrait apporter à notre application finale
	
\end{document}