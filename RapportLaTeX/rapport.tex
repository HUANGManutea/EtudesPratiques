\documentclass[a4paper,11pt]{article}

\usepackage{exptech}
\linespread{1,6}

%%%%%%%%%%%%%%%%%%%%%%%%%%%%%%%%%%%%%%%%%%%%%%%%%%%%%%%%%%%%%%%%%%%%%%%%%%%%%%%

\title{ \textbf{Création d'un modèle 3D à partir de dessins 2D} }
% Pour avoir le titre de l'expose sur chaque page

\author{ Aurélien \textsc{FONTAINE} Etienne \textsc{GEANTET} \\
	Manutea \textsc{HUANG} Arnaud \textsc{MARTIN} \\
	\\
	Encadrants : François \textsc{LEHERICEY}	Bertrand \textsc{COUASNON}}

\date{1 Juin 2015}                    % Ne pas modifier

%%%%%%%%%%%%%%%%%%%%%%%%%%%%%%%%%%%%%%%%%%%%%%%%%%%%%%%%%%%%%%%%%%%%%%%%%%%%%%%

\begin{document}

\maketitle                 % Génère le titre
\thispagestyle{empty}      % Supprime le numéro de page sur la 1re page

\begin{abstract}
	Lors de leur troisième année, les étudiants de l'INSA de Rennes doivent réaliser un projet. Notre groupe a choisi de travailler sur le développement d'une application sur tablette. Cette application doit permettre à un utilisateur novice de créer rapidement et facilement des objets en 3D à partir d'un dessin 2D. 
\end{abstract}
	
	\section{Remerciements}
		Nous souhaitons tout d'abord remercier Monsieur François LEHERICEY pour sa disponibilité ainsi que pour ses précieuses informations.
		
		Nous voudrions également remercier Monsieur Bertrand COUASNON pour son apport de connaissances ainsi que son aide dans l'élaboration de nos différentes présentations.
				
	\section{Introduction} %Pas oublier de changer GRIBOUILLI
		%Présentation succinte du projet
		Notre équipe s'est mobilisée autour d'une question : Comment un utilisateur novice peut-il créer facilement et rapidement un objet en 3D à partir d'un gribouilli ?
		
		Afin d'y répondre nous avons cherché à développer une application intuitive. Mais nous avions quelques contraintes : l'utilisateur dessinera sur tablette et les objets 3D devront être exportés sur un serveur Unity. Cet envoi à un serveur Unity doit permettre aux chercheurs de IRISA d'insérer rapidement des objets simples dans une de leurs scènes.
	\section{État de l'art}
		%Rappel rapide des technologies utilisées
	\section{Organisation du travail}
	%découpage des tâches et répartition du travail + explication séances hebdomadaires
	Dès le début du projet, nous nous sommes divisé le travail afin que chacun puisse avancer sans pour autant devoir attendre le travail d'un autre membre de notre équipe. Nous avons repéré les axes centraux du développement qui sont : l'affichage de la fenêtre de l'application, la capture et l'affichage du dessin fait par l'utilisateur, l'extrusion de la forme dessinée, et l'export de la figure finale sur un serveur Unity. 
	
	Afin de faire un point régulier et d'être sûr de répondre aux exigences de notre encadrant, nous nous sommes réunis hebdomadairement. Au cours de ces séances, nous pouvions ainsi voir le travail effectués par les différents membres de notre projet et nous pouvions être épaulés dans le développement par M. François LEHERICEY.
	\section{Présentation de l'application}
		%combien de temps passé pour chaque tâche
		%Problèmes rencontrées et solutions apportées, en quoi ces solutions ont été utilisées par la suite
		\subsection{L'IHM}
		\subsection{Le dessin}
			\subsubsection{Les outils}
			\subsubsection{La zone de dessin}
		\subsection{L'extrusion}
		\subsection{Placement de la figure}
		\subsection{La gestion de la liste des figures}
		\subsection{L'envoi}
	\section{Objectifs}
		%Nos objectifs ont-il été atteints? Si Non, quels ont été les choix faits en cours de routes pour changer ou abandonner certains objectifs.
	\section{Conclusion}
		%Ce que l'on se souviendra du projet
		%Si c'était à refaire, quoi changer?
		%Quelles sont les évolutions que l'on pourrait apporter à notre application finale
	
\end{document}