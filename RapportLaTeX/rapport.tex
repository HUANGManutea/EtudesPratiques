\documentclass[a4paper,11pt]{article}

\usepackage{exptech}
\linespread{1,6}

%%%%%%%%%%%%%%%%%%%%%%%%%%%%%%%%%%%%%%%%%%%%%%%%%%%%%%%%%%%%%%%%%%%%%%%%%%%%%%%

\title{ \textbf{Création d'un modèle 3D à partir de dessins 2D} }
% Pour avoir le titre de l'expose sur chaque page

\author{ Aurélien \textsc{FONTAINE} Etienne \textsc{GEANTET} \\
	Manutea \textsc{HUANG} Arnaud \textsc{MARTIN} \\
	\\
	Encadrants : François \textsc{LEHERICEY}	Bertrand \textsc{COUASNON}}

\date{4 Mai 2015}                    % Ne pas modifier

%%%%%%%%%%%%%%%%%%%%%%%%%%%%%%%%%%%%%%%%%%%%%%%%%%%%%%%%%%%%%%%%%%%%%%%%%%%%%%%

\begin{document}

\maketitle                 % Génère le titre
\thispagestyle{empty}      % Supprime le numéro de page sur la 1re page

\begin{abstract}
	Lors de leur troisième année, les étudiants de l'INSA de Rennes doivent réaliser un projet. Notre groupe a choisi de travailler sur le développement d'une application sur tablette. Cette application doit permettre à un utilisateur novice de créer rapidement et facilement des objets en 3D à partir d'un dessin 2D. 
\end{abstract}
	
	\section{Remerciements}
		Nous souhaitons tout d'abord remercier Monsieur François LEHERICEY pour sa disponibilité ainsi que pour ses précieuses informations.
		
		Nous voudrions également remercier Monsieur Bertrand COUASNON pour son apport de connaissances ainsi que son aide dans l'élaboration de nos différentes présentations.
				
	\section{Introduction} %Pas oublier de changer GRIBOUILLI
		%Présentation succinte du projet
		Notre équipe s'est mobilisée autour d'une question : Comment un utilisateur novice peut-il créer facilement et rapidement un objet en 3D à partir d'un gribouilli ?
		
		Afin d'y répondre nous avons cherchés à développer une application intuitive. Mais nous avions quelques contraintes : l'utilisateur dessinera sur la tablette et les objets 3D devront être exportés vers un serveur Unity. Cet envoi à un serveur Unity doit permettre aux chercheurs de l'IRISA d'insérer rapidement des objets simples dans une de leurs scènes.
	\section{État de l'art}
		%Rappel rapide des technologies utilisées
	\section{Organisation du travail}
	%découpage des tâches et répartition du travail + explication séances hebdomadaires
	Dès le début du projet, nous nous sommes divisés le travail et nous avons hébergé notre projet sous GitHub afin que chacun puisse avancer sans pour autant devoir attendre le travail d'un autre membre de notre équipe. Nous avons repéré les axes centraux du développement qui sont : l'affichage de la fenêtre de l'application, la capture et l'affichage du dessin fait par l'utilisateur, l'extrusion de la forme dessinée, le placement de celle-ci, et enfin l'exportation de la figure finale sur un serveur Unity. Ceux-ci ont donc été partagés entre nous.
	
	Afin de faire un point régulier et d'être sûr de répondre aux exigences de notre encadrant, nous nous sommes réunis hebdomadairement. Au cours de ces séances, nous pouvions ainsi voir le travail effectué par les différents membres de notre projet et nous pouvions bénéficier de l'expérience de M. François LEHERICEY avec Unity pour les problèmes dont nous ne trouvions pas les solutions.
	\section{Présentation de l'application}
		%combien de temps passé pour chaque tâche
		%Problèmes rencontrées et solutions apportées, en quoi ces solutions ont été utilisées par la suite
		\subsection{L'IHM}
			Nous devions d'abord créer le design de l'Interface Homme-Machine (IHM). Nous avons défini sur papier à quoi devaient ressembler les divers menus. Afin de les mettre en place sous Unity, nous avons dû nous servir de leur mise à jour 4.6 qui contient un système de fenêtres adapté à la création de menus (UI), sortie fin novembre 2014. Au vu de sa date de sortie récente, nous avions accès à peu d'aides sur Internet. Après une avancée assez laborieuse sur le début, nous avons réussi à mettre en place le canevas principal.
			%expliquer ce qu'est un canvas ?
			Le temps passé sur cette partie, nous avons permis d'acquérir les connaissances nécessaires sur les outils UI d'Unity et donc d'avancer plus rapidement dans la suite du développement. De plus les différents boutons étaient en place et n'attendaient plus que les scripts écrits dans les autres parties.
		
		\subsection{Le dessin}
			La capture du dessin étant une partie lourde, nous avons décidé de séparer les outils de cette partie. C'est la personne en charge de l'IHM, donc déjà accoutumée avec les menus qui s'en sont occupés.
		
		
			\subsubsection{Les outils}				
				Suite directe de l'IHM, un sous-menu coulissant devait être présent pour pouvoir choisir les outils de dessin. Pour rendre le menu coulissant, la solution qui nous a paru la plus adaptée est de se servir de la gestion des animations d'Unity. Une fois celles-ci faites, elles ont été liées à des boutons pour les enclencher.
					
				Dans cette partie la difficulté était de choisir comment coder la sélection d'outils. Pour cela, nous avons choisi de créer une énumération pour chaque ensemble d'outils (couleurs, diamètres, pinceau/gomme/pot de peinture). Des fonctions pour récupérer les outils courants ont été implantées, pour permettre au reste de l'application d'avoir connaissance de l'état des outils. De même des fonctions de Set\footnote{Une fonction de Set est un mutateur. Les mutateurs permettent de changer l'état de données membres tout en vérifiant si la valeur que l'on veut donner à la donnée membre respecte les normes de celle-ci ou diverses règles de cohérence.\emph{Wikipedia}} ont été liées aux boutons appropriés. Pour respecter les droits d'images, nous avons dessiné nous-même les icônes.
				
				Cette partie n'était pas très compliquée en elle-même. Après des tests utilisateurs, nous avons accéléré les animations. Et au cours du développement, ces animations ont dû être plusieurs fois refaites, car elles déclenchaient des bugs non expliqués.
			\subsubsection{La zone de dessin}
			
			C'est pour nous la première étape de la création d'un objet 3D. À la suite de cette étape, nous souhaitions obtenir une image à extruder. Les images sont appelées textures sous Unity. Nous avons utilisé un plan comme objet de référence pour la zone de dessin. Nous nous sommes rendus compte plus tard que la texture de base ne prenait pas l'alpha\footnote{C'est à dire la transparence}, or nous souhaitions disposer d'une image où seul les parties coloriées seraient extrudées. Nous avons donc précisé, lors de la création de la texture, vouloir un type RGB32, pour avoir un fond transparent.
			
			Un premier problème apparut alors quant au dimensionnement de cette zone de dessin. Le plan ayant une taille précise, il fallait être capable de modifier sa taille par rapport à celle du canevas. Nous avons donc rédigé un script qui initialise la dimension de la zone de dessin, autant pour le plan que pour la texture qui lui est associée.
			
			Il fallait ensuite être capable de dessiner sur la texture, or l'objet (le plan ici) et la texture ont des dimensions différentes : nous avons d'abord tenté de modifier la texture à l'endroit on l'ont cliquait, mais il y avait toujours un décalage à l'affichage. Finalement, la solution la plus efficace fut de projeter un Ray, objet monodirectionnel qui traverse l'espace, au moment où l'on clique dans la zone de dessin, et de trouver le pixel à l'intersection de ces deux objets. On a ainsi pu, dans un premier temps, modifier un pixel en cliquant dessus. Ensuite, pour garder un tracé continu, nous avons implémenté une méthode qui trace la droite entre le dernier point colorié, et le point sur lequel on clique. Il restait à faire le lien entre la zone de dessin, et les différents outils définis plus tôt. Ainsi, la zone de dessin va chercher les différentes couleurs, la largeur du trait, et les outils à utiliser dans le menu déroulant. Nous avons aussi implémenté des méthodes "eraser" et "bucket", capables de gommer et remplir rapidement une surface.
		\subsection{L'extrusion}
		
		L'extrusion est une autre tâche importante du projet. En effet, il faut partir d'une texture, créée un objet en 3D. Pour ce faire, nous sommes partis d'une application déjà existante : UCLA Mesh Creator. C'est un créateur d'objets 3D à partir d'une texture. Seulement, il a fallu repris le code pour l'adapter à notre utilisation. Initialement, cette application sert à créer rapidement et simplement un objet afin de pouvoir l'insérer dans son jeu rapidement. De plus, il est possible de la faire fonctionner uniquement en mode "édition". Il y a deux modes quand on l'on travaille sous Unity : le mode "édition", qui nous permet de créer des scènes avec nos objets, nos animations, ..., et un mode "jeu" où l'on voit le rendu final que l'utilisateur verra. Nous nous sommes donc concentrés sur l'adaptation du code déjà existant plutôt que sur la recherche et l'implémentation d'un ou plusieurs algorithmes afin de détecter les contours d'une image. Seulement, comme c'est une adaptation, il y a une longue phase de compréhension et de test afin de faire fonctionner comme il faut cette application. 
		\subsection{Placement de la figure}
		
			Pour placer une figure dans le décor 3D, il fallait créer des menus pour diriger celle-ci. Et intégrer une caméra qui filmerait l'objet courant selon ses axes. Après de multiples recherches sur la gestion des caméras Unity, nous avons finalement réussi à écrire un script assurant ce suivi avec une possibilité de changer aisément l'axe de vue. Ensuite il y a eu la mise en place d'un menu pour diriger notre figure. Celui-ci est relativement simple à mettre en place aux vues de ce que l'on a réalisé précédemment dans l'IHM. La difficulté de cette partie est de concevoir un script pour modifier la taille de notre figure.

		\subsection{La gestion de la liste des figures}
		\subsection{L'envoi}
	\section{Objectifs}
		%Nos objectifs ont-il été atteints? Si Non, quels ont été les choix faits en cours de routes pour changer ou abandonner certains objectifs.
	\section{Conclusion}
		%Ce que l'on se souviendra du projet
		%Si c'était à refaire, quoi changer?
		%Quelles sont les évolutions que l'on pourrait apporter à notre application finale
			Nous en retenons qu'Unity est un outil puissant, mais pas encore adapté à la création de menus. Ce travail nous a permis d'appréhender Git, ce qui est nécessaire dans notre domaine. Ce qui en ressort aussi c'est que l'on doit se fixer des objectifs clairs et essayer au maximum de se donner des délais pour chacune des tâches. Cela évite qu'une partie minime du projet ne nous accapare tout notre temps.
			
			Si l'on nous représentait ce projet, nous changerions notre méthode de travail sur plusieurs points. Notamment, nous passerions moins de temps sur l'état de l'art, dès le début il était quasiment sûr qu'Unity serait notre choix. Ensuite, nous procéderions à un partage des recherches et des réflexions aux premiers instants du projet. Et nous commencerions à apprendre Unity en octobre. Car ce manque d'apprentissage nous a fait défaut, nous cherchions des solutions à des problèmes précis sans nécessairement comprendre les solutions trouvées. Mais le manque de temps ne nous permettais pas de faire autrement en décembre.
			
			Notre application a encore plusieurs voies d'améliorations possibles. Comme l'implantation d'outils pour faire des sphères, avec des extrusions circulaires.
	
\end{document}