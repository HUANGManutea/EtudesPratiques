\documentclass[a4paper,11pt]{article}
\bibliographystyle{unsrt} % Le style est mis entre accolades.
\usepackage{exptech,hyperref}
\hypersetup{
	colorlinks=true,                         
	citecolor=black, % Couleur des numéros de la biblio dans le corps
	urlcolor=blue,   % Couleur des url
	linkcolor=black}  % Couleur des liens internes

%Décommanter pour la relecture (interlignes plus importantes)
%\linespread{1,6}

%%%%%%%%%%%%%%%%%%%%%%%%%%%%%%%%%%%%%%%%%%%%%%%%%%%%%%%%%%%%%%%%%%%%%%%%%%%%%%%

\title{ \textbf{Création d'un modèle 3D à partir de dessins 2D} }
% Pour avoir le titre de l'expose sur chaque page

\author{ Aurélien \textsc{FONTAINE} Etienne \textsc{GEANTET} \\
	Manutea \textsc{HUANG} Arnaud \textsc{MARTIN} \\
	\\
	Encadrants : François \textsc{LEHERICEY}	Bertrand \textsc{COUASNON}}

\date{4 Mai 2015}                    % Ne pas modifier

%%%%%%%%%%%%%%%%%%%%%%%%%%%%%%%%%%%%%%%%%%%%%%%%%%%%%%%%%%%%%%%%%%%%%%%%%%%%%%%

\begin{document}

\maketitle                 % Génère le titre
\thispagestyle{empty}      % Supprime le numéro de page sur la 1re page

\begin{abstract}
	En projet de troisième année, notre groupe a choisi de travailler sur le développement d'une application sur tablette. Cette application est destinée à être utilisée lors de séances de démonstration dans la salle de réalité augmentée Immersia. Nous devons permettre au sujet de la démonstration de réaliser simplement (et sans besoin d'explication sur le fonctionnement de l'application) des objets 3D à partir de dessins 2D qu'il a dessiné à main levée. Une fois sa création effectuée, elle doit être envoyée vers le serveur gérant la salle afin d'y afficher l'objet 3D.
\end{abstract}
	
	\section{Introduction} %Etienne
		%Présentation succinte du projet.	
		%ajouter beaucoup plus de détails en premer lieu.
		%Dire: Application qui tourne sur un serveur Unity à l'Irisa, ils veulent faire des objets 3D à intégrer dans leur scène
		Ce projet nous a été proposé par l'IRISA, avec pour but de répondre à la problématique suivante : Comment meubler rapidement et facilement, une scène de réalité virtuelle avec des objets 3D ? Notre application cherche ainsi à répondre à ce besoin en permettant une création d'objets 3D suffisamment intuitive pour être employée par un utilisateur novice. Notre application devra à terme, servir à meubler une scène sans perdre de temps à créer des objets avec des logiciels plus compliqués.
		
		Au cours de ce rapport, vous trouverez les choix faits pour développer cette application. Tout d'abord, la sélection des technologies les plus à même de respecter le cahier des charges. Ensuite comment à partir de ces choix, nous avons défini l'architecture de l'application. Sur chaque partie de celle-ci, vous verrez les choix réalisés par l'équipe. Puis nous présenterons notre répartition du travail et le suivit de celui-ci au sein de notre équipe. Pour finir, nous concluons sur les objectifs atteints et les futurs possibles pour le développement de cette application.

	\section{Cahier des charges} %Etienne
	%Ajouter des trucs, décrire l'application : dessin, extruder, assembler.
		Notre client nous a imposé le cahier des charges qui suit:
		\begin{itemize}
			\item Supportable par Android Tablette, l'utilisation du tactile permet de rendre la démonstration plus intuitive et interactive.
			\item Application ergonomique: utilisation rapide pour un nouvel utilisateur. Les icônes doivent donc être le plus clair possibles, l'utilisateur peut donc trouver ce qu'il veut rapidement. Et que celui-ci n'ai pas d'options à sélectionné.
			\item Outils de dessin: c'est à dire mettre à disposition une éventails de couleurs, la possibilité de changer le rayon de son crayon et plusieurs outils à disposition, comme une gomme par exemple.
			\item Création d'objet 3D à partir les dessins ainsi faits: l'application doit détecté les contours du dessin. Crée un objets 3D à partir de cette détection, et ensuite permettre à l'utilisateur d'assembler son objet avec ce qu'il a fait précédemment.
			\item Pouvoir exporter la création vers un serveur Unity: cette exportation doit de se faire sans fils et permettra ensuite de visualiser l'objet dans la salle de réalité augmentée.
		\end{itemize}

	\section{État de l'art} %Aurélien
	%Rappel rapide des technologies utilisées
	%Ajouter un petit texte avant la première sous-partie pour expliquer.
	%	\subsection{Quelle technologie utiliser ?}
		La première étape du projet fut d'établir un état de l'art, afin de déterminer quelles technologies utiliser, parmi celles que nous avons à disposition. Nous avons choisi Unity plutôt que d'autres moteurs de jeu comme Cry Engine ou Unreal Engine. C'était le choix le plus adapté à la gestion des objets dans une scène, et pour une gestion grandement simplifiée du transfert de l'objet vers le serveur, créé avec Unity. Nous aurions aussi pu nous tourner vers certains logiciels de dessin gérant la création d'objets 3D, tels Markers, LayerPaint ou SketchBook, mais nous nous sommes rendu compte que ceux-ci étaient souvent complexes à utiliser, ce qui entrait en conflit avec notre idée de faire une application simple et intuitive.
		
		En développant notre application sur Unity\cite{Unity}, nous avions accès à de nombreux outils libres pour nous aider. Nous avons ainsi décidé de nous servir de l'application UCLA Mesh Creator\cite{UCLA}, qui permet de créer un objet 3D à partir d'une image. Cela nous donnait une base solide pour toute \hyperlink{ancre}{la partie \ref{extrusion}, qui est l'extrusion}, développée plus loin à la page \pageref{extrusion}. 
		
	
		
	\section{Conception}
	
		Une fois la technologie sélectionné, nous avons dû établir précisément comment notre application allais fonctionner et donc dés maintenant définir quel serai l'approche la plus ergonomique du problème.
		
		\subsection{Comment être ergonomique?} %Arnaud
		Pour le dessin, nous avons créé une interface simplifiée dans l'esprit de Microsoft Paint. Avec un menu déroulant pour accéder aux outils (à gauche, le menu est fermé, à droite, ouvert):
		%ajouter espaces entre les captures.
		\centerline{\includegraphics[scale=0.5]{images/Cmt_placer2.png}}
		
		Une fois le dessin extrudé, pour le placer dans l'environnement 3D, l'utilisateur pourra à l'aide d'un menu (1) orienter et redimensionner la figure selon chacun de ses axes les uns après les autres. Pour se retrouver dans l'environnement, il aura une caméra qui suit l'objet selon l'axe courant (2). De plus, il y a une caméra amovible et zoomable qui filmera l'ensemble de la scène déjà créée (3). Ici les cubes sont des objets créés précédemment qui sont passés en transparent pour faciliter la lisibilité.
		\newline
		
		Une fois tout ces points définis , nous avons pu passer à l'étape suivante, c'est-à-dire l'architecture du logiciel.
		
		\subsection{Architecture logicielle} %Manutea
		Nous avons découpé le logiciel en plusieurs sous-parties qui peuvent être développées indépendamment. Les liens entre les parties obligent à créer des fonctions qui envoient certaines données. L'avantage de cette découpe est qu'une étape est en majorité indépendante de la précédente. Donc, le développement et la phase de test chaque partie peuvent se faire de façon optimisée avant de tout réunir et lier.
		
		Nous avons omis dans le schéma la caméra de prévisualisation, elle est contenu dans 'Objet', qui représente l'assemblage des divers dessins 2D extrudés.
		%Unity basé composant => pas d'héritage toussa toussa...
				\centerline{\includegraphics[scale=0.5]{images/archi.png}}


	\section{Etape par étape}
	%changer le nom de cette partie.
		%combien de temps passé pour chaque tâche
		%Problèmes rencontrées et solutions apportées, en quoi ces solutions ont été utilisées par la suite
		\subsection{L'IHM} %Arnaud
		%parler des modifications apportées après les tests soumis à d'autres gens
			La difficulté de la création de l'Interface Homme-Machine (IHM) était de concevoir un espace ergonomique et adapté pour recevoir le reste des éléments de l'application. Après une conception sur papier pour assurer cette adaptabilité, nous avons développé cette interface avec des outils récemment intégrés à Unity: les 'UI'. Ils sont dédiés à la création de fenêtres et de menus. Leur utilisation limite la production de codes de notre part et permet une gestion plus optimisée des éléments de l'interface.
			
			Voici donc l'IHM finale, avec une zone qui servira pour le dessin et qui servira pour le placement (1), une barre d'outils (2), une zone pour la prévisualisation (3) et une dernière pour la possible future gestion des figures créées (4).
			
			\centerline{\includegraphics[scale=0.5]{images/ihm.png}}

			Chaque bouton ouvre sur des fenêtres qui n'attendent plus que la liaison avec les fonctions des autres sections de l'application.
			
			Pour respecter la propriété intellectuelle, nous avons dessiné une partie des boutons et pris une autre sur iconmonstr.com\cite{iconmonstr} (un site proposant des icônes libres pour notre utilisation).
		\subsection{Dessin}
		
			C'est pour nous la première étape de création d'un objet 3D. L'utilisateur doit ici être capable de faire un dessin à main levée qui servira de support à l'extrusion. Le dessin comprend tout d'abord les différents outils à la disposition du dessinateur, et la zone où créer le dessin.
			
			\subsubsection{Outils}	%Arnaud
						
				Les outils sont dans un menu coulissant, pour animer le mouvement, un outil est intégré dans Unity, les animations. Nous avons ensuite accéléré ces animations après des tests utilisateurs.
					
				Pour permettre à un utilisateur novice d'utiliser l'application, nous avons proposé à l'utilisateur un nombre limité d'outils, de couleurs et de diamètres. Nous aurions pu être plus exhaustif, avec par exemple un choix des quantités de chaque couleur primaire entre 1 et 255 pour couvrir toute la palette chromatique. Mais cela aurait trop complexifié l'application.
				Dans cette partie il est très facile d'ajouter des outils ou des couleurs sans toucher au reste du programme. Il n'y a qu'une fonction à écrire et le canevas \footnote{Zone graphique permettant de dessiner. \emph{Wikipedia}} de ces fonctions est déjà présent dans le code.
			\subsubsection{Zone dessin} %Etienne
			À la suite de cette première étape de dessin, nous souhaitions obtenir une image à extruder. Les images sont appelées textures sous Unity. Nous avons utilisé un plan comme objet de référence pour la zone de dessin.
			
			Un premier problème est apparut alors quant au dimensionnement de cette zone de dessin. Le plan ayant une taille précise, il fallait être capable de modifier sa taille par rapport à celle du canevas. Nous avons donc rédigé un script qui initialise la dimension de la zone de dessin, autant pour le plan que pour la texture qui lui est associée.
			
			Le dessin se trace sur la texture à l'endroit où l'on clique. Or nous ne pouvions pas juste utiliser la position du curseur, car les dimensions de la texture sont indépendantes et s'adaptent de celle de l'objet. Un moyen efficace pour modifier la texture en cliquant sur l'objet est de projeter un Ray, objet monodirectionnel qui traverse l'espace, au moment où l'on clique dans la zone de dessin, et de trouver le pixel à l'intersection de ces deux objets. Nous avons ainsi pu, dans un premier temps, modifier un pixel en cliquant dessus. Ensuite, pour garder un tracé continu, nous avons implémenté une méthode qui trace la droite entre le dernier point colorié, et le point sur lequel on clique. Pour cela, nous utilisons l'équation cartésienne de la droite :
			\centerline{\includegraphics[scale=0.6]{images/trait.png}}
			Ensuite, nous observons les pixels contenus dans ce cadre augmenté de la largeur du trait pour ne pas en perdre dans les coins, ou quand le trait est quasi droit. Les pixels dont la distance à la droite est inférieure à la largeur du trait sont coloriés.
			
			 Il restait à faire le lien entre la zone de dessin, et les différents outils définis plus tôt. Ainsi, la zone de dessin va chercher les différentes couleurs, la largeur du trait, et les outils à utiliser dans le menu déroulant. Nous avons aussi implémenté des méthodes "eraser" et "bucket", capables de gommer et remplir rapidement une surface. Pour le bucket, nous avons utilisé un algorithme classique de remplissage par diffusion avec une pile explicite\footnote{Voir l'article sur \href{http://fr.wikipedia.org/wiki/Algorithme_de_remplissage_par_diffusion}{Wikipedia}}.
			
		\hypertarget{ancre}{
		\subsection{Passage au 3D}} %Aurélien
		\label{extrusion}
			L'extrusion est une autre tâche importante du projet. En effet, il faut à partir d'une texture, créer un objet en 3D. Pour ce faire, nous sommes partis d'une extension Unity déjà existante : UCLA Mesh Creator. C'est un créateur d'objets 3D, à partir d'une texture.
			
			L'utilisation de cet outil nous a permis un très gros gain de temps et de code. En effet, toute la partie concernant la détection des contours de l'image été déjà faite et fonctionnelle. Il nous a suffit d'intégrer ce code à notre application.
			
			L'adaptation du code existant à notre projet ne s'est pas faite naturellement car il y a eu une phase de compréhension du code existant et de sa modification. Il s'est avéré au cours du développement, que de nombreuses parties, notamment celles gérant l'acquisition de la texture à extruder, ne pouvait pas se faire simplement. Nous avons donc du reconstruire toute une partie afin de faire le fonctionner.
			
			Malgré ces contre-temps, l'utilisation de cet outil nous a été très bénéfique et il nous a surement permis de finir le projet dans les temps.
			%Expliquer le problème avec ce qui compile pas à cause de l'UnityEditor.
		\subsection{Placement et prévisualisation} %Arnaud
		
			Pour placer l'objet courant, nous avons juste intégré des scripts qui translatent la figure et des sliders\footnote{Un slider, ou curseur de défilement, est un composant d'interface graphique permettant d'entrer une valeur numérique dans un programme en déplaçant un curseur sur une échelle graduée. Dans certains cas, l'utilisateur peut spécifier une valeur simplement en cliquant sur un point de l’échelle. \emph{Wikipedia}} pour choisir la taille. La vitesse et rotation/translation et les tailles admises par les figures sont facilement modifiables, elles nécessiteront probablement des ajustements. Ou sinon un traitement de l'objet dans le serveur peut être imaginer, où celui-ci augmenterait la taille de l'objet pour l'adapter à la salle de réalité augmentée.
			
			Pour la prévisualisation et le placement, nous avons intégré des caméras. Celles-ci sont adaptées aux objets créés. Mais si on ajuste la taille des objets, il faudrait aussi modifier la taille de ces caméras.

		\subsection{Export} %Manutea
			Un fois l'objet en 3D créé, nous devons l'envoyer à un serveur qui affichera cet objet dans une scène 3D.
			Ce serveur est mis en place par l'IRISA et tourne sous Unity, nous avons donc jugé intéressant d'exploiter des solutions implémentées dans Unity ou en C\#.
			Nous nous sommes donc orientés vers le socket TCP qui permet d'échanger des données entre 2 applications.
			Nous avons choisi cette approche de l'envoi des données car :
			\begin{itemize}
				\item Elle dispose de classes prévues à cet effet en C\#
				\item Elle est documentée en C\# par Microsoft\cite{MSDN} 
				\item Il existe de nombreux tutoriels sur internet\cite{CodeProject}
				\item Elle permet d'envoyer des octets, donc flexible
			\end{itemize}
						
			
			\centerline{\includegraphics[scale=1]{images/tcp-socket.png}}
			
			Dès que la connexion est établie, le client envoie sous forme de flux les données de l'objet en 3D. Cette action correspond à la requête sur le schéma. Puis, en guise de traitement, le serveur affiche l'objet en 3D dans la scène.
			Dans notre cas, un script léger du côté du serveur permet de gérer la connexion avec le client et l'affichage de l'objet.
			
	\section{Organisation du travail}
				
				Pour réalisé toutes ces tâches en parallèles, nous avons hébergé notre projet sous GitHub. La répartition des tâches fut comme suis:
					\begin{itemize}

			\item	 Aurélien Fontaine : Passage au 3D
			\item	 Etienne Geantet : Zone Dessin
			\item	 Manutea Huang : Export
			\item	 Arnaud Martin : IHM - Outils - Placement - Prévisualisation
							
					\end{itemize}
				Nous n'avons pas rencontré de problème majeur lors de la mise en commun des parties. A part au niveau du passage de la texture de la zone de dessin à l'extrusion:
				%je vous laisse expliquer
				% 
				%
				%Etienne, Aurélien
				 Nous nous sommes rendus compte que la texture de base ne prenait pas l'alpha\footnote{C'est à dire la transparence} en compte, or nous souhaitions disposer d'une image où seules les parties coloriées seraient extrudées. Nous avons donc précisé, lors de la création de la texture, vouloir un type RGB32, pour avoir un fond transparent.
				%
				%
				%
				Afin de faire un point régulier et d'être sûr de répondre aux exigences de notre encadrant, nous nous sommes réunis hebdomadairement. Au cours de ces séances, nous pouvions ainsi voir le travail effectué par les différents membres de notre projet et nous pouvions bénéficier de l'expérience de M.François LEHERICEY avec Unity.
				
				Cette organisation du travail s'est majoritairement révélée efficace. Nous aurions dû fixer notre choix sur Unity dès octobre et passer moins de temps sur l'état de l'art pour le choix du logiciel. Cela nous aurait permis de faire des recherches plus ciblées et ainsi trouver UCLA Mesh Creator plus tôt. De plus, nous aurions pu commencer à acquérir des connaissances basiques sur Unity. Manque de connaissances qui nous a ralenti dans le début du développement.
			
	
	\section{Objectifs} 
		%Qu'est-ce qu'on a : changé, abandonné ...

	%	Pour le dessin, nous nous sommes contenté de trois tailles pour l'outil de coloriage.
	%	La possibilité de sauvegarder des objets n'était pas, à l'origine, une des fonctionnalités de l'application. Mais en se concertant, nous nous sommes rendu compte qu'une personne aimerait sûrement pouvoir réutiliser ses objets dans la construction d'autres objets. Si toutes les autres fonctionnalités de l'application sont opérationnelles, la sauvegarde serait un objectif potentiel à rajouter.
	%	Globalement, nous sommes arrivés à un résultat très proche de notre conception initiale de l'application.Notre application a encore plusieurs voies d'améliorations possibles. Comme l'implantation d'outils pour faire des sphères, avec des extrusions circulaires.
	%
	%
	%		On y met dans les deux parties autoures. Pas vraiment besoin de garder.
		Au cours de l'avancement du projet, nous avons du faire certains choix, et changer ou abandonner certaines de nos idées initiales. Notamment : la sauvegarde, ainsi que l'ouverture et l'édition des éléments créés.
		Par manque de temps, nous n'avons pas pu les implanter complètement. Par contre, dans l'IHM, des menus sont déjà présents pour gérer la sauvegarde et nous avons créé les débuts de scripts correspondant. Toutes ces informations seront intégrées dans la documentation utilisateur.
		Globalement, nous sommes arrivés à un résultat très proche de notre conception initiale de l'application et qui répond au cahier des charges.
	
	\section{Conclusion} %Aurélien
		%Ce que l'on se souviendra du projet
		%Si c'était à refaire, quoi changer?
		%Quelles sont les évolutions que l'on pourrait apporter à notre application finale
		%	Nous en retenons qu'Unity est un outil puissant, mais pas encore adapté à la création de menus, et orienté pair-à-pair. Ce travail fut aussi une bonne initiation à la gestion de projet, avec l'utilisation de Git. Il en ressort aussi qu'il est crucial de se donner des tâches précises et des délais à respecter pour arriver au bout d'un projet dans les temps impartis. Cela évite qu'une partie minime du projet ne nous accapare tout notre temps.
			
		%	Si l'on nous présentait ce projet à nouveau, nous changerions notre méthode de travail sur plusieurs points. Notamment, nous passerions moins de temps sur l'état de l'art; il était presque sûr qu'Unity serait notre choix comme outil de développement. Ensuite, nous procéderions à un partage des recherches et des réflexions aux premiers instants du projet. Nous commencerions à apprendre à utiliser Unity dès le mois d'Octobre, car ce manque de connaissances nous a relenti par la suite. Nous étions parfois bloqués par des détails, ou nous trouvions des solutions sans les comprendre entièrement, menant à d'autres problèmes par la suite. Mais le manque de temps ne nous permettait pas de faire autrement.
		Notre application respecte le cahier des charges imposé :
		L'application est donc utilisable sur tablette Windows et possède une interface graphique claire. Le dessin se fait de façons très intuitives, avec une utilisation similaire aux logiciels de dessin classiques. L'extrusion permet une prévisualisation, puis un placement guidé en trois étapes. Il est ensuite possible d'exporter l'objet créé vers un serveur. L'utilisation d'Unity, un logiciel puissant et évoluant, assure un suivi et de possibles évolutions facilitées par l'application.
		
		Nous avons réalisé des tests avec des utilisateurs, ce qui assure un certain niveau d'ergonomie. En précisant juste où se trouve les outils et le menu extrudés, ils arrivent à créer des objets 3D sans rencontrer de problème.
		
		L'utilisation d'un système de gestion de version est un vrai plus dans un projet de groupe, nous permettant une répartition efficace du travail. Il faut aussi garder à l'esprit que nous réfléchissons tous différemment: il est donc primordial de bien définir les règles qui permettent de lier les différentes fonctionnalités de l'application.
		

	
	\section{Remerciements}
		Nous souhaitons tout d'abord remercier Monsieur François LEHERICEY pour sa disponibilité ainsi que pour ses précieuses informations.
		
		Nous voudrions également remercier Monsieur Bertrand COUASNON pour son apport de connaissances ainsi que son aide dans l'élaboration de nos différentes présentations.
		

\bibliography{bibli} % mon fichier de base de données s'appelle bibli.bib

\end{document}