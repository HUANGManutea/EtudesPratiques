\documentclass[a4paper,11pt]{article}

\usepackage{exptech}
\linespread{1,6}

%%%%%%%%%%%%%%%%%%%%%%%%%%%%%%%%%%%%%%%%%%%%%%%%%%%%%%%%%%%%%%%%%%%%%%%%%%%%%%%

\title{ \textbf{Création d'un modèle 3D à partir de dessins 2D} }
% Pour avoir le titre de l'expose sur chaque page

\author{ Manutea \textsc{HUANG} Aurélien \textsc{FONTAINE} \\
	Arnaud \textsc{MARTIN} Etienne \textsc{GEANTET} \\
	\\
	Encadrants : François \textsc{LEHERICEY}	Bertrand \textsc{COUASNON}}

\date{1 Juin 2015}                    % Ne pas modifier

%%%%%%%%%%%%%%%%%%%%%%%%%%%%%%%%%%%%%%%%%%%%%%%%%%%%%%%%%%%%%%%%%%%%%%%%%%%%%%%

\begin{document}

\maketitle                 % Génère le titre
\thispagestyle{empty}      % Supprime le numéro de page sur la 1re page

\begin{abstract}
	Lors de leur troisième année, les étudiants de l'INSA de Rennes doivent réaliser un projet. Notre groupe a choisi de travailler sur le développement d'une application sur tablette. Cette application doit permettre à un utilisateur novice de créer rapidement et facilement des objets en 3D à partir d'un dessin 2D. 
\end{abstract}
	
	\section{Remerciements}
		Nous souhaitons tout d'abord remercier Monsieur François LEHERICEY pour sa disponibilité ainsi que pour ses informations précieuses.
		
		Nous voudrions également remercier Monsieur Bertrand COUASNON pour son apport de connaissances ainsi que son aide dans l'élaboration de nos différentes présentations.
				
	\section{Introduction}
	%Pas oublier l'état de l'art !
	\section{Documentation utilisateur}
		\subsection{Le dessin}
		\subsection{L'extrusion - la placement de la nouvelle forme}
		\subsection{L'envoi}
			
	\section{Documentation technique}
	
\end{document}