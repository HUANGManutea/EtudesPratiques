\documentclass[a4paper,11pt]{article}

\usepackage{exptech,hyperref}
\hypersetup{
	colorlinks=true,                         
	citecolor=black, % Couleur des numéros de la biblio dans le corps
	urlcolor=blue,   % Couleur des url
	linkcolor=black}  % Couleur des liens internes

%Décommanter pour la relecture (interlignes plus importantes)
%\linespread{1,6}

%%%%%%%%%%%%%%%%%%%%%%%%%%%%%%%%%%%%%%%%%%%%%%%%%%%%%%%%%%%%%%%%%%%%%%%%%%%%%%%

\title{ \textbf{Création d'un modèle 3D à partir de dessins 2D Documentation Utilisateur} }
% Pour avoir le titre de l'expose sur chaque page

\author{ Aurélien \textsc{FONTAINE} Etienne \textsc{GEANTET} \\
	Manutea \textsc{HUANG} Arnaud \textsc{MARTIN} \\
	\\
	Encadrants : François \textsc{LEHERICEY}	Bertrand \textsc{COUASNON}}

\date{4 Mai 2015}                    % Ne pas modifier

%%%%%%%%%%%%%%%%%%%%%%%%%%%%%%%%%%%%%%%%%%%%%%%%%%%%%%%%%%%%%%%%%%%%%%%%%%%%%%%

\begin{document}

\maketitle                 % Génère le titre
\thispagestyle{empty}      % Supprime le numéro de page sur la 1re page

\tableofcontents


\section{Dessin}
	Lors du lancement de l'application, une zone de dessin (zone en rouge sur l'image ci-dessous) apparaît. 
	\includegraphics[scale=0.6]{./images/zonedessin.png}
	
	Pour sélectionner un outil de dessin, appuyez sur le menu déroulant (en vert ci-dessus). Vous pouvez alors choisir parmi les outils suivants :
	\includegraphics[scale=0.6]{./images/img7.png}
	A gauche, vous pouvez sélectionner la couleur.
	A droite, les outils (sur fond bleu) et la largeur du trait (points sur fond gris).
	
	
\section{Extrusion et placement}
%en une ou deux sections, à voir...	

\section{Export}
	Il y a deux étapes à suivre dans l'ordre pour la partie export.
	\begin{enumerate}
		\item Un personnel de l'IRISA renseigne l'adresse IP du serveur
		\item L'utilisateur exporte l'objet final
	\end{enumerate}
	\subsection{IRISA}
		\begin{enumerate}
		\item Cliquez sur le bouton \includegraphics[scale=0.7]{./images/engrenage.png} en haut à gauche pour afficher les options.
		
		\item Entrez l'adresse IP dans le champ disponible.
		
		\item Pour valider, appuyez sur ENTRER ou cliquez n'importe où.
		
		\item Fermez la fenêtre à l'aide du bouton \includegraphics{./images/closewindow.png}
		\end{enumerate}
	\subsection{Utilisateur}
			Une fois que l'objet est terminé, cliquez sur le bouton \includegraphics[scale=0.8]{./images/export.png}
		
\end{document}