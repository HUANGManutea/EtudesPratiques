\documentclass[a4paper,10pt]{beamer}
\usepackage[utf8x]{inputenc}
\usepackage[T1]{fontenc}
\usepackage[french]{babel}
\usepackage{hyperref,graphicx,multicol,eurosym,tabularx,color}
\usetheme{Berkeley}
\setbeamercolor{structure}{fg=cyan!60!black}
\setbeamertemplate{navigation symbols}{\large \insertframenumber /\inserttotalframenumber}
\newcolumntype{M}[1]{>{\centering\arraybackslash}m{#1}}

\title{Création d'objets 3D à partir de dessins 2D}
\author[Groupe 3INFO]{Aurélien Fontaine, Manutea Huang,
\\ Etienne Geantet, Arnaud Martin}
\institute[INSA de Rennes]{Institut National des Sciences Appliquées de Rennes}
\date{\today}

\begin{document}
	
	\begin{frame}
		\begin{titlepage}
			\centerline{\includegraphics[scale=0.1]{images/logos/logoINSA.jpg}}
			\centerline{Encadrants : François Lehericey and Bertrand Coüasnon}	
		\end{titlepage}
	\end{frame}
	
	\section{Introduction}
	
	\begin{frame}
		\begin{itemize}
		\item Un projet proposé par les chercheurs de l'Irisa.
		\end{itemize}
		\centerline{\includegraphics[scale=0.25]{images/intro/Immersia.jpg}}
		\centerline{Comment meubler rapidement une scène de réalité}
		\centerline{virtuelle avec divers objets 3D ?}
	\end{frame}
	
	\begin{frame}
		Pour l'étude du comportement physique des modèles 3D dans une scène, les chercheurs souhaitent pouvoir créer des objets rapidement.
		
		
		\begin{itemize}
			\item Ce projet a pour but de permettre la \textbf{création rapide} d'objets simples.
			\centerline{\includegraphics[scale=0.5]{images/intro/car.jpg}}
			\item L'application met l'accent sur la \textbf{simplicité} d'utilisation et \textbf{l'ergonomie}.
		\end{itemize}
		
	\end{frame}
	
	\begin{frame}
		\tableofcontents
	\end{frame}
	
	\section{Cahier des charges}
	
	\begin{frame}
		Cahier des charges Etienne 
	\end{frame}
	
	\section{Etat de l'art}
	
	\begin{frame}
		Quelle techno Aurélien
	\end{frame}
		

	
	\section{Conception}	
		\begin{frame}{Conception}
	 		En parallèle du choix de technologie, nous avons dû réfléchir à sur deux points centraux:
		
			\begin{itemize}
				  \item Définir comment être ergonomique : découpé pas à pas pour s'assurer de la simplicité à chaque instant
				  \item Divisé le logiciel en éléments les plus simples possibles et indépendant pour organisé le développement
			\end{itemize}
		\end{frame}
		
		
		\begin{frame}{Comment être ergonomique?}
				Repérage des parties pouvant posé problèmes:
				
				\begin{itemize}
					\item Comment dessiner?
					\item Comment extruder ce dessin?
					\item Comment placer la figure créée dans l'environnement 3D?
					\item Comment se repérer dans notre création?
				\end{itemize}
		\end{frame}	
		
		\begin{frame}{Comment être ergonomique?}

				Comment dessiner?
					\begin{itemize}
						\item Interface sobre
						\item Menu d'outils déroulant
					\end{itemize}
				
				\centerline{\includegraphics[scale=0.3]{images/Nono/img1.png}} \centerline{\includegraphics[scale=0.3]{images/Nono/img2.png}}
			


		\end{frame}	
		
		\begin{frame}{Comment être ergonomique?}
			
			Comment extruder ce dessin?
			\begin{itemize}
				\item Jeu d'icônes assez clair
				\item Limité les options
				\item L'utilisateur n'as pas à intervenir plus que nécessaire
			\end{itemize}
			
			\centerline{\includegraphics[scale=0.5]{images/Nono/img3.png}} 
			
			
			
		\end{frame}	
			
		\begin{frame}{Comment être ergonomique?}
						
				Comment placer la figure créée dans l'environnement 3D?
				
				\begin{itemize}
					\item Découpé en plusieurs étapes
					\item Menu clair
					\item Ne pas surcharger d'informations
				\end{itemize}
				
				\centerline{\includegraphics[scale=0.5]{images/Nono/img4.png}} 
						
						
						
		\end{frame}	
		
		\begin{frame}{Comment être ergonomique?}
			
			Comment se repérer dans notre création?
			
			\begin{itemize}
				\item Vue d'ensemble permanente
				\item Caméra mobile
				\item Limité les déplacements
			\end{itemize}
			
			\centerline{\includegraphics[scale=0.5]{images/Nono/img5.png}} 
			
			
			
		\end{frame}	
	\begin{frame}{Architecture logicielle} %Découpage de l'application}
		\huge Comment découper l'application?
	\end{frame}
	
	\begin{frame}{Architecture logicielle} %Découpage de l'application}
		%Archi logi Manutea
		\centerline{\includegraphics[scale=0.3]{images/archilogi/archi1.png}}
	\end{frame}
	\begin{frame}{Architecture logicielle} %Découpage de l'application}
		%Archi logi Manutea
		\centerline{\includegraphics[scale=0.3]{images/archilogi/archi2.png}}
	\end{frame}
	\begin{frame}{Architecture logicielle} %Découpage de l'application}
		%Archi logi Manutea
		\centerline{\includegraphics[scale=0.3]{images/archilogi/archi3.png}}
	\end{frame}
	\begin{frame}{Architecture logicielle} %Découpage de l'application}
		%Archi logi Manutea
		\centerline{\includegraphics[scale=0.3]{images/archilogi/archi4.png}}
	\end{frame}
	\begin{frame}{Architecture logicielle} %Découpage de l'application}
		%Archi logi Manutea
		\centerline{\includegraphics[scale=0.3]{images/archilogi/archi5.png}}
	\end{frame}	
	\begin{frame}{Architecture logicielle} %Découpage de l'application}
		%Archi logi Manutea
		\centerline{\includegraphics[scale=0.3]{images/archilogi/archi6.png}}
	\end{frame}
			
	\section{Etape par étape}
		%Expliquer nos choix par rapports aux objectifs énoncés
			\begin{frame}{Etape par étape}
				Ce qui a été fait:
					\begin{itemize}
						\item Choix technologique : Unity
						\item Définition du fonctionnement de l'application
						\item L'architecture du logiciel
					\end{itemize}
				Grâce tout cela, nous avons les éléments pour développer l'application.
			\end{frame}
		
		
	\begin{frame}{IHM}
			Problématique:
				\begin{itemize}
					\item Accueillir les éléments définis précédemment
					\item Simple
				\end{itemize}
				
			Solution:
				\begin{itemize}
					\item Technologie: utilisation du système d'UI (outils de menus et fenêtre) d'Unity
					\item Des espaces prêt à accueillir les autres composants
				\end{itemize}
				\centerline{\includegraphics[scale=0.3]{images/Nono/img6.png}} 
	\end{frame}
	
	\begin{frame}{Menu d'outils}
		Pour assurer l'ergonomie et l'évolution de ce menu:
			\begin{itemize}
				\item Choix de couleur limité
				\item Choix de taille limité
				\item Choix d'outils limité
					\centerline{\includegraphics[scale=0.4]{images/Nono/img7.png}} 
				\item Modèle de fonction simple
							\centerline{\includegraphics[scale=0.6]{images/Nono/img8.png}} 
			\end{itemize}


	\end{frame}
	
	\begin{frame}
		Dessin -> zone Etienne
	\end{frame}
	
	\begin{frame}
		Aurélien Passage au 3D
	\end{frame}
	
	\begin{frame}{Placement}
			Modèle défini précédemment:
					\centerline{\includegraphics[scale=0.4]{images/Nono/img4.png}} 
					Il faut:
			\begin{itemize}
				\item Une caméra de suivi et change d'axe
				\item Des boutons pour déplacé l'objet
				\item Pouvoir modifié la taille de l'objet. Choix retenu : taille globale et taille selon un axe
				\item Rendre les autres objets transparents

			\end{itemize}
					
	\end{frame}
	
	\subsection{Envoi des données}
	
	\begin{frame}{Envoi des données}
		\centerline{\includegraphics[height=120pt]{images/network/sending_model2.png}}
	\end{frame}
	
	
	\begin{frame}{Côté serveur}
		\centerline{\includegraphics[height=150pt]{images/network/plugin1.png}}
		\invisible{
		\begin{itemize}	
			\item  Léger 
			\item Reçoit les données 
			\item Intégré dans la scène 3D
			\item Affiche l'objet 
		\end{itemize}	}
		
	\end{frame}
	\begin{frame}{Côté serveur}
		\centerline{\includegraphics[height=150pt]{images/network/plugin.png}}
		\begin{itemize}	
			\item \pause Léger \pause
			\item Reçoit les données \pause
			\item Affiche l'objet 
		\end{itemize}	
		
	\end{frame}
	
	
	\begin{frame}{Socket TCP}
				\centerline{\includegraphics[height=150pt]{images/network/tcp-socket1.png}}
				\invisible{
					\begin{itemize}
						\item Dispose de classes prévues à cet effet en C\#
						\item Documentée en C\# par Microsoft
						\item Permet d'envoyer des octets, donc flexible
					\end{itemize}
				}
	\end{frame}
	
	\begin{frame}{Socket TCP}
		\centerline{\includegraphics[height=150pt]{images/network/tcp-socket2.png}}
		\invisible{
			\begin{itemize}
				\item Dispose de classes prévues à cet effet en C\#
				\item Documentée en C\# par Microsoft
				\item Permet d'envoyer des octets, donc flexible
			\end{itemize}
		}
	\end{frame}
	\begin{frame}{Socket TCP}
		\centerline{\includegraphics[height=150pt]{images/network/tcp-socket3.png}}
		\invisible{
			\begin{itemize}
				\item Dispose de classes prévues à cet effet en C\#
				\item Documentée en C\# par Microsoft
				\item Permet d'envoyer des octets, donc flexible
			\end{itemize}
		}
	\end{frame}
	\begin{frame}{Socket TCP}
		\centerline{\includegraphics[height=150pt]{images/network/tcp-socket4.png}}
		\invisible{
			\begin{itemize}
				\item Dispose de classes prévues à cet effet en C\#
				\item Documentée en C\# par Microsoft
				\item Permet d'envoyer des octets, donc flexible
			\end{itemize}
		}
	\end{frame}
	\begin{frame}{Socket TCP}
		\centerline{\includegraphics[height=150pt]{images/network/tcp-socket5.png}}
		\invisible{
			\begin{itemize}
				\item Dispose de classes prévues à cet effet en C\#
				\item Documentée en C\# par Microsoft
				\item Permet d'envoyer des octets, donc flexible
			\end{itemize}
		}
	\end{frame}
	\begin{frame}{Socket TCP}
		\centerline{\includegraphics[height=150pt]{images/network/tcp-socket6.png}}
		\invisible{
			\begin{itemize}
				\item Dispose de classes prévues à cet effet en C\#
				\item Documentée en C\# par Microsoft
				\item Permet d'envoyer des octets, donc flexible
			\end{itemize}
		}
	\end{frame}
	\begin{frame}{Socket TCP}
		\centerline{\includegraphics[height=150pt]{images/network/tcp-socket6.png}}
			\begin{itemize}
				\item Dispose de classes prévues à cet effet en C\# \pause
				\item Documentée en C\# par Microsoft \pause
				\item Permet d'envoyer des octets, donc flexible
			\end{itemize}
	\end{frame}
	
	\begin{frame}{Côté client}
		\centerline{\includegraphics<1>[height=140pt]{images/network/options.png}}
	\end{frame}
	
	\begin{frame}{Côté client}
		\centerline{\includegraphics<1>[height=140pt]{images/network/polandball_3D.png}}
	\end{frame}
	
	\begin{frame}{Résultat}
		\centerline{\includegraphics<1>[height=170pt]{images/network/server.png}}
	\end{frame}
	
	\section{Conclusion}
	
	\begin{frame}
		Aurélien Conclu
		%vidéo -> présente les points clés de notre présentation
	\end{frame}
		
\end{document}
